\documentclass[10pt]{article}
\usepackage[margin=1in]{geometry}
\usepackage{times}
\usepackage{graphicx}
\usepackage{amsmath}
\usepackage{booktabs}
\graphicspath{{../docs/figures/}{../../docs/figures/}{../docs/}{../../docs/}}

\title{Low-data Fine-tuning for Personalized Cognitive Agents}
\author{Gustave Vernay \\ Avisia / Aivancity / \'{E}cole Polytechnique (candidate PhD)}
\date{}

\begin{document}
\maketitle

\begin{abstract}
We investigate low-data fine-tuning strategies (LoRA/QLoRA/P-tuning v2) for adapting LLMs to a single user under data scarcity. We evaluate alignment, forgetting, and efficiency on synthetic persona datasets.
\end{abstract}

\section{Introduction}
% TODO: Frame personalization need under privacy and scarcity constraints.
\subsection{Context}
Personal agents require individualized behavior, yet user data is limited and sensitive.

\subsection{Problem}
Full fine-tuning is compute-heavy and risks overfitting/forgetting under low-data regimes.

\subsection{Contributions}
\begin{itemize}
	\item A benchmark of light-weight methods (LoRA, QLoRA, P-tuning v2) for tiny-persona settings.
	\item An evaluation protocol measuring alignment, catastrophic forgetting, and compute budget.
	\item Analysis of trade-offs and ablations (rank, token budget).
\end{itemize}

% Figure: FT full vs lightweight adapters (placeholder)
% \begin{figure}[t]
%   \centering
%   \includegraphics[width=0.9\linewidth]{pb_full_vs_adapters.pdf} % expected at docs/figures/pb_full_vs_adapters.pdf
%   \caption{Comparison of full fine-tuning vs lightweight adapters (LoRA/QLoRA/Prefix).}
%   \label{fig:full-vs-adapters}
% \end{figure}

\section{Background}
\subsection{Parameter-Efficient Fine-Tuning}
LoRA and QLoRA reduce trainable parameters and memory via low-rank updates and quantization.

\subsection{Prefix/Prompt Tuning}
P-tuning v2 and prefix-tuning optimize soft prompts for task specialization.

\subsection{Continual Learning and Personalization}
We relate PEFT to continual learning for maintaining prior capabilities.

% Table: Technical characteristics
\begin{table}[t]
	\centering
	\caption{Technical characteristics of PEFT methods: trained parameters, VRAM, and speed.}
	\label{tab:tech}
	% Placeholder table for PEFT technical characteristics (PB)
\begin{tabular}{lrrr}
\toprule
Method & Trainable Params (M) & VRAM (GB) & Relative Speed \\
\midrule
Full FT & 7000 & 80+ & 1.0x \\
LoRA & 10--50 & 16--24 & 2.0x \\
QLoRA & 10--50 & 8--16 & 2.5x \\
P-tuning v2 & 1--5 & 8--16 & 3.0x \\
\bottomrule
\end{tabular}

\end{table}

\section{Method}
\subsection{Dataset}
We consider a tiny-persona corpus (\textless{}50k tokens) describing preferences and goals.

\subsection{Tasks}
QA, preference adherence, and refusal out-of-scope.

\subsection{Metrics}
Alignment score (judge LLM), catastrophic forgetting index, and compute budget (time, VRAM).

% Figure: Experimental protocol (placeholder)
% \begin{figure}[t]
%   \centering
%   \includegraphics[width=0.9\linewidth]{pb_protocol.pdf} % expected at docs/figures/pb_protocol.pdf
%   \caption{Train → eval → judge pipeline for low-data FT.}
%   \label{fig:protocol}
% \end{figure}

\section{Experiments}
\subsection{Models and Baselines}
We evaluate Mistral 7B, LLaMA 3B, and baselines including zero-shot and prompt-only.

\subsection{Ablations}
LoRA rank and token budget sensitivity analyses.

% Table: Dataset statistics
\begin{table}[t]
	\centering
	\caption{Dataset statistics for PB: number of personas, prompts, and token counts.}
	\label{tab:dataset-pb}
	% Placeholder table for PB dataset statistics
\begin{tabular}{lrrr}
\toprule
Split & Personas & Prompts & Tokens (k) \\
\midrule
Train & 50 & 2,000 & 30 \\
Dev & 10 & 400 & 6 \\
Test & 15 & 600 & 9 \\
\bottomrule
\end{tabular}

\end{table}

\section{Results}
\subsection{Main Comparisons}
LoRA offers stable gains; QLoRA is memory-efficient under 4-bit quantization.

\subsection{Forgetting and Rehearsal}
Rehearsal reduces forgetting while preserving alignment.

% Figures: performance vs dataset size; alignment/forgetting bars (placeholders)
% \begin{figure}[t]
%   \centering
%   \includegraphics[width=0.85\linewidth]{pb_perf_vs_size.pdf} % expected at docs/figures/pb_perf_vs_size.pdf
%   \caption{Performance vs. dataset size under different PEFT methods.}
%   \label{fig:perf-vs-size}
% \end{figure}
% \begin{figure}[t]
%   \centering
%   \includegraphics[width=0.85\linewidth]{pb_alignment_forgetting.pdf} % expected at docs/figures/pb_alignment_forgetting.pdf
%   \caption{Alignment and forgetting across methods.}
%   \label{fig:align-forget}
% \end{figure}

\section{Discussion}
\subsection{Trade-offs}
Analyze efficiency vs robustness in low-data regimes; discuss privacy (local FT feasibility).

\subsection{Limitations}
Multimodal transfer and dependence on data quality.

\section{Conclusion}
Low-data FT is viable; future work includes continual learning and multimodal adaptation.

\bibliographystyle{plain}
\bibliography{refs}
\end{document}