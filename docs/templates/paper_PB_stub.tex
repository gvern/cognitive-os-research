\documentclass{article}
\usepackage{neurips_2025}
\usepackage{times}
\usepackage{graphicx}
\usepackage{amsmath}

	itle{Low-data Fine-tuning for Personalized Cognitive Agents}

\author{
	Gustave Vernay \\
	Avisia / Aivancity / École Polytechnique (candidate PhD) \\
		exttt{gustave.vernay@example.com} \\
}

\begin{document}

\maketitle

\begin{abstract}
We investigate low-data fine-tuning strategies to adapt large language 
models to individual users with limited data (under 50k tokens). We compare 
LoRA, QLoRA, and P-tuning v2 on synthetic persona datasets and evaluate 
alignment, forgetting, and efficiency. Results show that lightweight adapters 
achieve stable personalization with minimal compute.
\end{abstract}

\section{Introduction}
- Problem: LLMs lack persistent personalization.  
- Challenge: user data is scarce and sensitive.  
- Contribution: systematic evaluation of low-data FT methods.  

\section{Background}
- LoRA/QLoRA.  
- P-tuning v2, prefix adapters.  
- Personalization benchmarks.  

\section{Method}
- Synthetic tiny persona dataset.  
- Task suite: QA, preference alignment.  
- Evaluation protocol (alignment, forgetting, compute budget).  

\section{Results}
- LoRA stable at 10k tokens, QLoRA efficient at 4-bit.  
- Forgetting mitigated with rehearsal prompts.  

\section{Conclusion}
- Low-data fine-tuning viable for Cognitive OS adaptation.  
- Future work: continual learning, cross-modal personalization.  

\bibliographystyle{plain}
\bibliography{refs}

\end{document}
%% Insert the prefilled LaTeX from the chat here.