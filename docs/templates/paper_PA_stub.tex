\documentclass{article}
\usepackage{neurips_2025} % ou iclr2025/chi selon soumission
\usepackage{times}
\usepackage{graphicx}
\usepackage{amsmath}

	itle{User Knowledge Graphs for Personal Cognitive Systems}

\author{
	Gustave Vernay \\
	Avisia / Aivancity / École Polytechnique (candidate PhD) \\
		exttt{gustave.vernay@example.com} \\
}

\begin{document}

\maketitle

\begin{abstract}
We propose a dynamic user knowledge graph (UKG) as a core representation 
for personal cognitive systems. Unlike static vector-based retrieval, UKGs 
encode entities, temporal relations, and provenance, supporting both robust 
retrieval and long-term consolidation. We present an incremental update 
mechanism and evaluate coherence, coverage, and groundedness.
\end{abstract}

\section{Introduction}
- Motivation: Need for coherent user memory representations.  
- Limitations of current RAG (hallucinations, lack of traceability).  
- Contribution: Dynamic UKG framework.  

\section{Related Work}
- Knowledge graphs (Wikidata, ConceptNet).  
- RAG and hybrid retrieval.  
- Cognitive architectures (ACT-R, SOAR).  

\section{Method}
- Schema design (entities, goals, projects).  
- Incremental updates (event sourcing, CRDT-like merges).  
- Consolidation cycles.  

\section{Experiments}
- Synthetic personal dataset (notes, tasks, calendar).  
- Baselines: vector-only, static KG.  
- Metrics: coherence, coverage, grounded QA.  

\section{Results}
- UKG improves groundedness by X\%.  
- Lower contradiction rate compared to vector-only.  

\section{Conclusion}
- UKG as foundation for Cognitive OS.  
- Future work: probabilistic UKGs, integration with fine-tuned LLMs.  

\bibliographystyle{plain}
\bibliography{refs}

\end{document}
%% Insert the prefilled LaTeX from the chat here.